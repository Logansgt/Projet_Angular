\documentclass[a4paper,11pt]{article} % Type de papier et taille police

\usepackage[T1]{fontenc}
\usepackage[french]{babel}
\usepackage{graphicx}
\usepackage{lmodern}
\usepackage[utf8]{inputenc}
\usepackage{parskip}
\usepackage{enumitem}  % énumération de points

\usepackage{listings} %pour afficher du code source
\usepackage{xcolor}

\usepackage{graphicx}%Insertion d'image

% Configuration du style du code
\lstset{
    language=Python,                % Langage par défaut
    basicstyle=\ttfamily\small,    % Police à chasse fixe
    stringstyle=\color{red},       % Couleur des chaînes
    keywordstyle=[1]\color{blue},     % Couleur des mots-clés de la catégorie 1
    keywordstyle=[2]\color{orange},    % Couleur des mots-clés de la catégorie 2
    keywordstyle=[3]\color{magenta},    % Couleur des mots-clés de la catégorie 3
    keywordstyle=[4]\color{green!60!black},    % Couleur des mots-clés de la catégorie 4
    commentstyle=\color{green!40!black}, % Couleur des commentaires
    numbers=left,                  % Numéros de ligne à gauche
    numberstyle=\tiny\color{gray}, % Style des numéros
    frame=single,                  % Cadre autour du code
    breaklines=true                % Retour à la ligne automatique
}


% Définition du javaScript pour les affichages

\lstdefinelanguage{TypeScript}{
  morekeywords=[1]{export, class, let, var, const, function, return,this}, % bleu
  morekeywords=[2]{string, number, Date, void, boolean}, % orange
  morekeywords=[3]{import, implements, interface, Component, selector,constructor, true, false}, % rose
  morekeywords=[4]{public,private}, %vert
  sensitive=true,
  morecomment=[l]{//},
  morecomment=[s]{/*}{*/},
  morestring=[b]",
  morestring=[b]'
}

%Info document
\title{Tutoriel d'utilisation du framework Angular}
\author{ Logan SAGET }
\date{30/01/2026}

\begin{document}



\maketitle % Affiche le titre défini plus haut
\tableofcontents % Génère la table des matières automatiquement
\newpage   % Saut de page

%Section numéro 1

\section{\underline{Initialisation d'un projet :}} %déclaration d'une section


\subsection{Commandes utiles à la création d'un projet (dans le dossier voulu) :} %déclaration d'un paragraphe


%lstlising est utilisé pour écrire du code, on peut préciser le langage
\begin{lstlisting} [language=bash]
    ng new [nomProjet]
    // Ou encore
    ng new [nomProjet] 
         --style=scss  // Langage du style
         --skip-tests=true

    cd nomProjet
    ng serve // Lancement du serveur
\end{lstlisting}

\newpage

%Section numéro 2


\section{\underline{Bases d'un projet Angular : }}

\subsection{Création et utilisation d'un composant :}

Génération automatique :

\begin{lstlisting} [language=bash]
    ng generate component page-accueil
\end{lstlisting}

Cette commande génère 3 fichiers :

%énumération de points
\begin{itemize}[label=\textbullet, left=1em]
  \item fichier HTML
  \item fichier de style
  \item fichier de script
\end{itemize}

Dans le fichier script on peut retrouver la classe qui défini ce composant :

\begin{lstlisting}[language=TypeScript]
import { Component } from '@angular/core';

@Component({
    selector: 'app-test',
    imports: [],
    templateUrl: './test.html',
    styleUrls: ['./test.scss'],
})

export class Test {

}   
\end{lstlisting}

On pourra bien sûr rajouter des lignes dans cette classe (comme des attributs par exemple).

\begin{lstlisting}[language=TypeScript]
export class Comp1{
  title!: string;
  description!: string;
  createdAt!: Date;
  snaps!: number;
}  
\end{lstlisting}

\newpage

Afin d'initialiser ces attributs (dès l'appel du composant), on implémente l'interface OnInit :

\begin{lstlisting}[language=TypeScript]
export class Comp1 implements OnInit{
  title!: string;
  description!: string;
  createdAt!: Date;
  snaps!: number;

  ngOnInit(): void {
    this.title = "premierComposant";
    this.description = "premierComposant";
    this.createdAt = new Date();
    this.snaps = 0;
  }
}
\end{lstlisting}

Une fois initialisée, on peut utiliser ces variables dans le fichier.html du composant :

\begin{lstlisting}[language=HTML]
<H2>{{title}}</H2>
<p>{{description}}</p>
\end{lstlisting}

Ce qui donnera : 
\begin{center} %centrer
    \fbox{\includegraphics[width=10cm]{images/rendu1Composant.png}}
\end{center}

Pour ajouter une image, on la défini d'abbord comme nouvel attribut, puis on utilise une balise src entre "[]" :

\begin{lstlisting}[language=HTML]
<img [src]="url"> <!-- url etant un attribut defini dans componant.js -->
\end{lstlisting}

\newpage
\subsection{Personnalisation des composants :}

Dans un premier lieu, on crée un package dans l'application dans lequel on va ajouter des classes :

\begin{center} %centrer
    \fbox{\includegraphics[width=10cm]{images/premiereClasse.png}}
\end{center}


Dans cette classe, on défini les attributs qui se trouvaient initialement dans le fichier ts du composant.

\begin{lstlisting}[language=TypeScript]
export class FaceSnap {
  constructor(public title: string,
              public description: string,
              public createdAt: Date,
              public snaps: number,
              public url: string) {}
      }
\end{lstlisting}

Dans le fichier "app.ts", nous allons maintenant déclarer des objets de type FaceSnap :

\begin{lstlisting}[language=TypeScript]
export class App implements OnInit {

  snap1!: FaceSnap;
  snap2!: FaceSnap;
  snap3!: FaceSnap;

  ngOnInit() {
    this.snap1 = new FaceSnap("premierComposant","premierComposant",new Date(),0,url);
    this.snap2 = new FaceSnap("2emeComposant","2emeComposant",new Date(),10,url);
    this.snap3 = new FaceSnap("3emeComposant","3emeComposant",new Date(),100,url);
  }
}
\end{lstlisting}

\newpage

Une fois la classe crée, on peut retirer les attributs du composant et simplement crée un objet de type FaceSnap :

\begin{lstlisting}[language=TypeScript]
export class Comp1 implements OnInit{
  @Input() faceSnap!: FaceSnap;
}
\end{lstlisting}

Enfin, on peut appeler nos composants dans l'html de l'application en initialisant l'attribut crée :

\begin{lstlisting}[language=HTML]
<app-comp1 [faceSnap]="snap1"></app-comp1>

<app-comp1 [faceSnap]="snap2"></app-comp1>

<app-comp1 [faceSnap]="snap3"></app-comp1>
\end{lstlisting}

Avec cette manipulation, on peut personnaliser nos composants.

Il ne faut pas oublier dans le html du composant de remplacer les appels avec faceSnap.attribut :

\begin{lstlisting}[language=HTML]
<H2>{{faceSnap.title}}</H2>
<p>{{faceSnap.description}}</p>
<img [src]="faceSnap.url">
\end{lstlisting}  

\newpage


%Section numéro 3

\section{\underline{Gestion des événements : }}

La gestion des événements est beaucou plus simple grâce à angular.
Il suffit de créer dans le TypeScript du composant des actions. Par exemple, on veut pouvoir liker un post et retirer le like :

\begin{lstlisting}[language=TypeScript]
export class Comp1 implements OnInit{
  @Input() faceSnap!: FaceSnap;

  snapButton!: string;
  snapOrNot!: boolean;

  ngOnInit(): void {
    this.snapOrNot = false;
    this.snapButton = "snaps"
  }

  onAddSnap() :void{
    if(this.snapOrNot === false){
      this.faceSnap.addSnap();
      this.snapOrNot = true;
      this.snapButton = "UnSnap";
    }else{
      this.faceSnap.delSnap();
      this.snapOrNot = false;
      this.snapButton = "snaps";
    }
  }
}
\end{lstlisting}

Enfin, on lie cet évènement à un bouton, ici, on parle de l'évènement "click", qu'on va donc écrire entre parenthèse dans la balise du bouton pour le définir :

\begin{lstlisting}[language=HTML]
<p>
  <button (click)="onAddSnap()">{{snapButton}}</button>
  {{faceSnap.snaps}}
</p>
\end{lstlisting}

\newpage


%Section numéro 4

\section{\underline{Mise en place d'un service :}}

\subsection{Création du service :}

La mise en place d'un service est utile afin de regrouper les méthodes qui seront utiles à plusieurs composants. On les utilise aussi afin de faire appel à des API.

On définit un service de la manière suivante :

\begin{lstlisting}[language=TypeScript]
    @Injectable({
  providedIn: 'root'
})
export class FaceSnapsService {
  private faceSnaps: FaceSnap[] = [
    new FaceSnap("premierComposant", "premierComposant", new Date(), 0)
    new FaceSnap("2emeComposant", "2emeComposant", new Date(), 10)
    new FaceSnap("3emeComposant", "3emeComposant", new Date(), 100)
  ];
}

\end{lstlisting}

\subsection{Utilisation d'un service :}

En utilisant la propriété injectable, on peut l'utiliser dans les différentes classes en l'injectant dans les constructeurs :


\begin{lstlisting}[language=TypeScript]

  Export class SingleFaceSnap{
   constructor(private snapsService: FaceSnapsService)
  }

\end{lstlisting}

Dans notre exemple, notre service possède des méthodes afin de trouver via un id un snap :
\newpage

\begin{lstlisting}[language=TypeScript]
    @Injectable({
  providedIn: 'root'
})
export class FaceSnapsService {
  private faceSnaps: FaceSnap[] = [
    new FaceSnap("premierComposant", "premierComposant", new Date(), 0)
    new FaceSnap("2emeComposant", "2emeComposant", new Date(), 10)
    new FaceSnap("3emeComposant", "3emeComposant", new Date(), 100)
  ];

    getSnapFaces(): FaceSnap[] {
      return [...this.faceSnaps] // Pour retourner un tableau independant meme si il a les memes references
  }

  getSnapById(id: string): FaceSnap {
    const trouverSnap = this.faceSnaps.find(FaceSnap => FaceSnap.id === id );
    if (!trouverSnap) {
      throw new Error("No snap found with ID " + id);
    }else{
      return trouverSnap;
    }
  }

  snapFaceById(faceSnapId: string, snapType: SnapType) {
      const trouverSnap = this.getSnapById(faceSnapId);
      trouverSnap.snap(snapType);
    }
}

\end{lstlisting}


Afin d'utiliser la méthode getSnapById, on va simplement injecté le service dans la classe ou l'on veut l'utiliser, puis l'appeler :


\begin{lstlisting}[language=TypeScript]

  export class SingleFaceSnap implements OnInit{

  constructor(private snapsService: FaceSnapsService) {
  }

  faceSnap!: FaceSnap;

  ngOnInit(): void { 
    this.faceSnap = this.snapsService.getSnapById(faceSnapId);
  }

  }

\end{lstlisting}

\section{\underline{Navigation et routes :}}

\subsection{Définition des routes :}

Dans le fichier routes de l'application, on retrouve un tableau de Routes. On va y insérer toutes les routes de notre application : 

\begin{lstlisting}[language=TypeScript]
  
export const routes: Routes = [
  {path: 'facesnaps', component: FaceSnapList}, // route qui va afficher le contenu de FaceSnapList
  {path: '', component: HubPage}, // route vide, donc le hub
];

\end{lstlisting}

Ensuite, dans notre app.ts, on va pouvoir importer "RouterOutlet".
Grâce à ceci, il sera possible dans le HTML de définir, au lieu d'un composant précis, l'objet qui se trouve à l'emplacement de la route dans l'url :


\begin{lstlisting}[language=HTML]
<app-header/>
<router-outlet/> // Le composant trouve a cette route
\end{lstlisting}

\subsection{Naviger entre les composants :}

Pour naviguer, il suffit d'utiliser routerLink (qu'on oublie pas d'importer dans le .ts de la où on l'utilise).
Par exemple, avec un composant header : 

\begin{lstlisting}[language=HTML]

<header>
  <p>Application</p>
  <nav>
    <button routerLink="" routerLinkActive="active" [routerLinkActiveOptions]="{exact: true}">HomePage</button>
    <button routerLink="facesnaps" routerLinkActive="active">App</button>
  </nav>
</header>

\end{lstlisting}

Ici, routerLinkActive sert à donné la classe active à l'élément dont la route est celle selectionnée.

\newpage

Maintenant, on peut naviguer dans la pages sans problème entre le hub et l'app : 

\begin{center}
    \fbox{\includegraphics[width=13cm]{images/renduAvecRoutes.png}}
\end{center}

\subsection{Accéder à un élément en particulier :}

Il est possible dans les routes de définir une variable qui changera : 

\begin{lstlisting}[language=TypeScript]
  
export const routes: Routes = [
  // La route avec l'id du snap afficher
  {path: 'facesnaps/:id', component: SingleFaceSnap},
  {path: 'facesnaps', component: FaceSnapList},
  {path: '', component: HubPage},
];

\end{lstlisting}

Grâce aux méthodes pour récupérer un snap via un id, on peut dans une classe qui reprends nos snap (copié collé de Comp1) faire :

\begin{lstlisting}[language=TypeScript]
  
constructor(private snapsService: FaceSnapsService,
  private route: ActivatedRoute) {
  }

  faceSnap!: FaceSnap;

  ngOnInit(): void {
    const faceSnapId = this.route.snapshot.params['id']; // Recup le face snap qui correspond a l'id dans l'url
    this.faceSnap = this.snapsService.getSnapById(faceSnapId); // On va recup le faceSnap qui correspond a cette id pour l'afficher
  }

\end{lstlisting}

Il ne manque plus qu'un moyen de mettre l'id du face snap dans l'url, j'utilise un bouton lié par un événement à cette méthode dans Comp1 : 


\begin{lstlisting}[language=TypeScript]
  
constructor(private router: Router){}

  @Input() faceSnap!: FaceSnap; // Accepte une valeur qui vient du parent

  protected onview() {
   this.router.navigateByUrl('facesnaps/'+this.faceSnap.id); // Methode pour inclure l'id choisi dans l'url
  }

\end{lstlisting}

On peut maintenant choisir quel composant regarder en cliquant sur le bouton view : 

\begin{center}
    \fbox{\includegraphics[width=13cm]{images/renduComposantParId.png}}
\end{center}

\end{document}